\documentclass[sigconf]{acmart}

% \usepackage{booktabs} % For formal tables

\usepackage{graphicx}
\usepackage{here} % [H]とするとその場所に配置されるらしい

% Copyright
%\setcopyright{none}
%\setcopyright{acmcopyright}
%\setcopyright{acmlicensed}
\setcopyright{rightsretained}
%\setcopyright{usgov}
%\setcopyright{usgovmixed}
%\setcopyright{cagov}
%\setcopyright{cagovmixed}


% % DOI
% \acmDOI{10.475/123_4}
% 
% % ISBN
% \acmISBN{123-4567-24-567/08/06}
% 
%Conference
\acmConference[AVI2018]{International Conference on Advanced Visual Interfaces}{May 2018}
              {Resort Riva del Sole, Castiglione della Pescaia, Grosseto (Italy)}
\acmYear{2018}
\copyrightyear{2018}
% 
% 
% \acmArticle{4}
% \acmPrice{15.00}

% These commands are optional
%\acmBooktitle{Transactions of the ACM Woodstock conference}
% \editor{Jennifer B. Sartor}
% \editor{Theo D'Hondt}
% \editor{Wolfgang De Meuter}


\begin{document}
\title{EpisoDAS - DAS-based password generation \\
using episodic memories}
% \titlenote{Produces the permission block, and copyright information}

\author{Toshiyuki Masui}
% \authornote{Is this necessary?.}
% \orcid{1234-5678-9012}
\affiliation{%
  \institution{Keio University}
  \streetaddress{5322 Endo}
  \city{Fujisawa}
  \state{Kanagawa}
  \postcode{252-8520}
}
\email{masui@masui.org}

\renewcommand{\shortauthors}{T. Masui}

\begin{CCSXML}
  <ccs2012>
  <concept>
  <concept_id>10002978.10002991.10002992.10011618</concept_id>
  <concept_desc>Security and privacy~Graphical / visual passwords</concept_desc>
  <concept_significance>500</concept_significance>
  </concept>
  </ccs2012>
\end{CCSXML}
\ccsdesc[500]{Security and privacy~Graphical / visual passwords}

\keywords{Passwords, Visual passwords, Authentication}

\begin{abstract}

We introduce a simple and powerful visual interaction technique for
managing strong passwords.
%
Passwords have been used for qauthentication for decades, but
appropriate handling of passwords is difficult because people can
easily forget passwords and they can be easily cracked.
%
To make the authentication process easier, various visual interaction
methods have been proposed, including the DAS (draw-a-secret)
method. Using DAS, users can log into various services just by drawing
a secret pattern on the screen.
%
Using a DAS-based authentication method, users can quickly log into a
service without typing a password. However, remembering complex secret
patterns can be as difficult as remembering passwords. We developed
EpisoDAS, with which users can generate strong passwords based on
their secret episodic memories with a simple DAS interface.
%
users can use secret patterns for authentication, based on their
secret episodic memories which they cannot easily forget.

\end{abstract}

\maketitle

\section{Introduction}

Passwords have been used as a means of authenticating to Web services
and applications for a long time, and remain the most popular
authentication method on the Internet.
Since short passwords are easily guessable by attackers and using the
same password for multiple services is unsafe, a different long
password should be used for each service a person uses.
However, remembering numerous long passwords is almost impossible for
ordinary humans.

However, password-based authentication is still the most convenient
and widely deployed method \cite{Bonneau:ReplacePasswords}, and is not
expected to disappear any time soon \cite{Herley:2009:PSS:1601990.1602010}.
%
To tackle this problem,
we have been proposing the \textit{EpisoPass} password generator with which 
users can generate strong passwords easily
based only on a user's secret and unforgettable episodic memories.

% For this reason, we believe that it is far better to ``generate''
% something for the authentication, based on a user's episodic
% memories. This has the benefit that unlike a password, a person is
% highly unlikely to forget such episodic memories.

\begin{figure}[H]
  \centerline{\includegraphics[width=12cm,bb=-300 -100 1222 1014]{figures/episopass.png}}
  \caption{EpisoPass}
  \label{EpisoPass}
\end{figure}

Password generation on EpisoPass is performed through the following steps:

\begin{enumerate}
\item A user registers multiple question texts related to their own personal
secret unforgettable episodic memories. For each question they must provides
a single correct answer and multiple additional incorrect answers.

\item The user provides a long ``seed string'' for each service that requires
a password.

\item EpisoPass shows the questions and answers to the user allowing
them to select the correct answer for each question.
Based on the user's selections,
EpisoPass substitutes characters in the seed string and generates a
strong password candidate string.
After selecting all of the correct answers,
the user copies the calculated string
and registers it as the password for the service.
\end{enumerate}

The biggest advantage of using EpisoPass is that
users don't have to remember strong password strings.
%
Users of EpisoPass can save the questions and answers
wherever is convenient. They can
then easily generate passwords by running
EpisoPass and answering the questions.
%
If a question is based on old unforgettable episodic memory,
there is little chance of losing the password for a service
as long as the seed string and questions are available.
%
Answering to many questions takes much longer time than typing a
remembered password, so people may prefer typing password
rather than using EpisoPass if they have to log in to a service frequently.


For the people who don't like to use passwords for authentication,
various types of authentication methods have been proposed.
Especially, visual authentication methods
like graphical passwords\cite{Biddle:2012:GPL:2333112.2333114,GraphicalPasswords}
and Draw-a-Secret (DAS) \cite{DAS} are promising approaches, because
people can remember visual patterns easier than passwords.

% \begin{figure}[H]
%   \includegraphics[width=8cm,bb=0 0 448 351]{figures/Dejavu.png}
%   \caption{D\'{e}j\`{a} Vu}
%   \label{fig:sample}
% \end{figure}

\begin{figure}
  % \includegraphics[width=8cm,bb=0 0 840 792]{figures/DAS.png}
  % \includegraphics[width=8cm,bb=0 0 320 321]{figures/DASexample.jpg}
  \includegraphics[width=12cm,bb=0 0 1000 667]{figures/AndroidLock.jpg}
  \caption{Pattern lock for Android.}
  \label{AndroidLock}
\end{figure}

Visual authentication methods have better characteristics than
password-based authentication, but
they are either more fragile to attacks
or remembering is difficult.

We propose a new visual password generation system \textit{EpisoDAS},
which have the advantages of both EpisoPass and DAS-based methods.

\section{EpisoDAS}

EpisoDAS is a visual authentication interface
based on the user's episodic memories.
Like EpisoPass,
EpisoDAS displays a list of questions to the user and
ask the user to choose the most appropriate answers.
EpisoDAS then generates a password string based on the user's selections.
%
% with which users can generate strong password based on their
% memory of visual patterns and episodic memories.
% Users can either select answers based on the user's secret episodic memories or
% draw a secret pattern represented by the pattern of selected answers.
% 
% Thus, users can recognize EpisoDas as a DAS system or
% use it as an implementation of EpisoPass.
%
The candidate answers are shown to the user as a grid of buttons,
and the user can select the most appropriate answer
by clicking one of the buttons in the grid.

If the series of correct answers form a special pattern
that the user can remember,
the user can select the series of correct answers quickly
just like using other DAS systems.

% http://episopass.com/Amazon_john@example.com/Amazon123456
% http://episopass.com/Amazon_john@example.com.html

\begin{figure}[H]
  \includegraphics[width=12cm,bb=0 0 1398 1060]{figures/EpisoDAS1.jpg}
  \caption{First question}
  \label{EpisoDAS1}
\end{figure}

Figure \ref{EpisoDAS1} shows
the initial screen of an EpisoDAS session.
% how a user can generate a password from his episodic memories.
%
If the user remembers that he had trouble at a MacDonald's in Boston in the past,
he would click ``Boston'', and the next question (Figure \ref{EpisoDAS2}) is displayed.
The contents and the order of the questions
do not depend on the user's actions, and
the user cannot tell whether he has selected a right answer.

\begin{figure}[H]
  \includegraphics[width=12cm,bb=0 0 1396 1062]{figures/EpisoDAS2.jpg}
  \caption{Second question}
  \label{EpisoDAS2}
\end{figure}

If he remembers that he had a bad time at Los Angeles
after reading the question in Figure \ref{EpisoDAS2},
he would click ``Los Angeles'' and proceed to the third question.
%
After repeating the slection process ten times,
the user can get his password
calculated from his selections. (Figure \ref{Result})

\begin{figure}[H]
  \includegraphics[width=12cm,bb=0 0 1160 468]{figures/result.png}
  \caption{Generated password}
  \label{result}
\end{figure}

If the series of right-answer buttons form a special DAS pattern,
% If the series of buttons are aligned next to each others,
the user can drag the pointing device like
Figure \ref{draw} and quickly get the generated
password without clicking buttons.
%
The candidate answers change according to the drag or click of the
pointing device, and the user can mix the dragging and clicking actions
at any point.

\begin{figure}[H]
  \includegraphics[width=12cm,bb=0 0 1130 1236]{figures/draw.png}
  \caption{Drawing a secret pattern}
  \label{draw}
\end{figure}

The drawing action by the user is close to the actions
used on conventional DAS systems.
Users can quickly generate his password for authentication
just by
drawing a pattern on the EpisoDAS screen.

EpisoDAS can be used for generating passwords for any kind of systems and
services, but it is most useful when used on a login window.
Figure \ref{Amazon} shows how a user can use EpisoDAS when
he wants to log in to Amazon.com.
If the EpisoDAS extension is installed on the browser,
an EpisoDAS window appears when the user enters his
email address and clicks the password field at Amazon.com.
After selecting answers, a password is generated and pasted
into the password field, making the user to log in without typing a password.

% Figure \ref{Amazon} shows how a user can use EpisoDAS on a login page,
% using a browser extension software.
% The extension shows a EpisoDAS window when a user tries to log into
% an Web service, and after the user selects the right answers,
% a password is generated from the input and pasted in to the
% password text box.

\begin{figure}[H]
  \includegraphics[width=12cm,bb=0 0 1092 1026]{figures/Amazon.png}
  \caption{Login to Amazon.com}
  \label{Amazon}
\end{figure}

The data used in the above example is retrieved from the EpisoDAS database
on the Web, using the name of the service (e.g. \texttt{Amazon})
and the user id (e.g. \texttt{joe@example.com}).

\section{Registration}

\subsection{Registering a DAS pattern}

EpisoDAS is the predecessor of the EpisoPass system,
and questions and answers should be registered on the EpisoPass system.
After registering questions and answer candidates,
a user can register a DAS pattern so that
answer candidates are easily selected on EpisoDAS.

\begin{figure}[H]
  \includegraphics[width=12cm,bb=0 0 1832 1180]{figures/JohnEpisodes.png}
  \caption{EpisoPass window}
  \label{JohnEpisodes}
\end{figure}

After selecting the correct answers
(``Boston'', ``Los Angeles'', etc.) on the EpisoPass page,
the user can click the ``DAS'' button on the top-right portion of the
EpisoPass page,
%
% After cliking the button,
% a page for registering the DAS pattern id shown to ther user, and
and the user can specify his DAS patten on the registration page.

\begin{figure}[H]
  \includegraphics[width=12cm,bb=0 0 1332 1118]{figures/DASRegister.png}
  \caption{Registring the DAS pattern}
  \label{DASRegister}
\end{figure}

If the user wants to use a U-shape as the secret pattern,
he can draw the shape shown in Figure \ref{draw}
on the registration window.
%
After finishing the registration drawing, the candidate answers fo the EpisoPass is
shuffled so that the correct answers represent the specified pattern.

\subsection{Registering questions and answers}

Questions and answer candidates can be edited interactively on the
EpisoPass page (Figure \ref{EpisoPass}).
However, creating many questions and answer candidates is hard because
people usually feel it hard to remember secret episodes.
%
To help people easily create EpisoPass problems,
we have provided a special page for easy problem creation
(Figure \ref{Easy}).

\begin{figure}[H]
  \includegraphics[width=12cm,bb=0 0 1678 1728]{figures/Easy.png}
  \caption{Easily creating problems}
  \label{Easy}
\end{figure}

First, users enter the names of their old friends and acquaintances.
After creating a list of such names,
the users are asked to enter special episodes of the people.
For example, if the user remembers that Tom was
very good at drawing, he can add a question
``Good painter?'' to the list, assuming Tom is the correct answer.

Recalling a special old episode is fairly hard for most people, but
recalling an attribute of an old friends seems to be much easier.

\section{Discussion}

The authors have been using EpisoPass for more than 5 years, and
using EpisoDAS for about an year.
No formal evaluation has been performed so far, but
important impressions have been observed during the period.

\subsection{Authentication speed}

Using EpisoDAS, very small amount of time is required for
the authentication process.
For a skilled computer user,
selecting answers based on the secret is a little bit slower than
typing a password on a computer with a hardware keyboard, but
EpisoDAS is faster on smartphones and tablet computers.

% 拡張機能を利用すると
% 記憶しているパスワード入力よりも高速なぐらいで入力できる

\subsection{Forgetting passwords}

The biggest advantage of using EpisoPass is that users don't have to
remember long password strings.

After the authors began using EpisoPass,
we have been having no problem forgetting passwords of any kind.

As long as a user have the seed string and question-answer data,
he can easily generate passwords by running
EpisoPass and answering the questions.

The seed string and question-answer data can be put on a public place
if they include enough amount of secrets.

If a question is based on old unforgettable episodic memory,
there is little chance of losing the password for a service
as long as the seed string and questions are available.

If the user's memory is related to an episode that took place 20 years
ago, say, and the user clearly still remembers the episode now, it is
unlikely the episode will be forgotten in the future.

\subsection{Universal}

EpisoDAS is implemented as a single HTML file that includes a JavaScript
script for password calculation.
We can put the HTML file on the Web or in a handy USB memory device
and access it from any Web browser.
If the questions are properly selected and nobody knows the correct answet to
the questions,
the data can be put on a public Web page or save it in a secure
repository like GitHub.
The data set of the authors are put on the Web, and it can be easily
found using Web search engines.
It means that the passwords used by the authors can be generated
even when the data is not kept locally.

\subsection{Security strength}

The strength of the generated passwords depends on the number and 
level of secrecy of the questions.
%
There have been many studies on the strength of passwords
\cite{Hayashi:2011:DSP:1978942.1979326,Komanduri:2011:PPM:1978942.1979321}, % What password is strong
but measuring the strength of secret questions is an area where more study is needed.
% \marginnote{I calculate $20^{10} = 10,240,000,000,000$. This is
% considerably more than a billion (closer to 10 thousand billion,
% (US) or 10 million million (UK).}
Using ten secret questions each with twenty answers, where each answer is considered equally likely,
an attacker would need to cover a search space of over ten trillion ($20^{10}$) values to check
every potential combination of answers,
and the entropy of the system is 43.2 ($10 \times \log_2 20$) bits.  % 10.0 * (Math.log(20.0)/Math.log(2.0))
%
This represents roughly the same entropy as using a random 8-character Latin-alphabet
password, for which the entropy is 45.6 bits.
This level is considered to be strong enough for Web services,
where it's not possible to perform online brute-force attacks \cite{Florencio:2007:SWP:1361419.1361429}.
  
\subsection{Creating questions}

The quality of the questions is key to using EpisoDAS effectively.  If
the episode is shared by someone else, this other person might easily
be able to answer the question and generate the user's password.
%
The episode related to each question should therefore not be known to any other person,
and the episode should be unforgettable.
%
Finding such episodes seems difficult at first,
but using the method shown in the last section,
it is not super-difficult to recall
many trivial episodes which are unforgettable but
nonetheless not important to other people.

Questions such as ``Who did you like best?'' should be avoided,
since friends or family might know the user's tastes, thereby allowing 
others to select the correct answer.
Also, memories related to taste is more fragile than memories of facts.

% \subsection{Question reuse}
% 
% 使い回しが駄目
% どこかのサイトのパスワードが流出し、
% そのEpisoPassデータが公開されていたら
% ブルートフォースで答がわかってしまうから
% 問題を使い回すことは好ましくない

\subsection{Browser extension}

EpisoDAS browser extension is very useful, but creating the extension
is cumbersome because there is no standard login form for
different Web services.
%
The browser extension should first realize that the Web page is a login page
of a Web service (e.g. Amazon),
and then it should be able to get the ID of the user
and know where the result of EpisoDAS calculation should be pasted.

\subsection{Sholder surfing}

Simple DAS authentication is known to be fragile for ``shoulder surfing''
attacks\cite{Aviv:2017:TBS:3134600.3134609}.
If the secret pattern is simple enough and
the pattern can be easily observed over the user's shoulder,
attackers can easily 再現する the login pattern.

This problem is common to all the DAS-based authentication systems,
and great care should be taken when drawing a secret in the public.
EpisoDAS users can use complex DAS patterns
because there is no risk of forgetting the pattern,
because the user can use EpisoDAS even when he does not remember the
secret pattern.

\subsection{Forgetting pattern risks}

Using a conventional DAS system, users should be carefull about
remembering the DAS pattern.
DAS patterns may be more easily remembered than password texts,
but there is always some risks about forgetting them.
Using EpisoDAS, users can easily remember the DAS pattern
just by reading the questions and answers.
Users can be very sure that they will never forget the pattern,
because the pattern can be retrieved by answering the questions.

EpisoDAS is the combination of EpisoPass and DAS-based authentication methods.
A user can generate a password using his episodic memories, or
he can generate a password using the DAS pattern he remembers.
%
EpisoDAS has the advantages of both of the systems: i.e.
users never forget how to generate the password, and
users can quickly generate his password.

\section{Conclusion}

\bibliographystyle{ACM-Reference-Format}
\bibliography{paper}

\end{document}
