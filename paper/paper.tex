\documentclass[twoside]{wiss}
%
%
% 問題を作るのが面倒だから、シードや回答を変えて運用するのがいい
%

%\usepackage{ascmac}

\usepackage{graphicx}
\usepackage{here} % [H]とするとその場所に配置されるらしい

\long\def\EP{\textsf{EpisoPass}}
\long\def\PW{パスワード}
\long\def\SS{シード文字列}
\long\def\EM{エピソード記憶}
\long\def\SQ{秘密の質問}

\journalhead{{\EP}: Holy Grail}

\begin{document}

\title{{\EP}: {\EM}にもとづく{\PW}管理}
%\etitle{{\EP}: The Ultimate Passwords Management System}
\etitle{{\EP}: Password Management based on Episodic Memories}

\author{増井俊之\affil{Toshiyuki Masui, 慶應義塾大学 環境情報学部}}

\begin{abstract}
忘却しにくい{\EM}にもとづく{\SQ}を使って強力な{\PW}を生成/管理するシステム「\textsf{\EP}」を提案する。
{\EP}は、ユーザの作成した{\SQ}への回答にもとづいて{\SS}を換字することによって強力な{\PW}を生成する。
{\SS}や回答のバリエーションにより異なる{\PW}が生成されるので様々なサービスに対して異なる{\PW}を生成できる。
既存の{\PW}を生成する{\SS}を利用することにより{\PW}の管理もできる。
適切な運用により、{\PW}に関連する情報を公開しても強力な{\PW}の生成/管理が可能である。
(★と言うと言い過ぎと言われるだろう)
\end{abstract}

\maketitle

\section{はじめに}

個人認証のために{\PW}が現在広く利用されている。
{\PW}認証には多くの問題があることが知られているが\cite{増井_ユニマガ}、
今後も長期にわたって利用され続けることが予想されるため、
適切に運用するための工夫が必要になっている。

{\PW}の長期的記憶が難しいことは{\PW}認証の大きな問題点のひとつである。
安全に運用するためには{\PW}はランダムで長い文字列であることが望ましいが、
そのようなものを頭の中に記憶しておくことは難しい。
また複数のサービスを利用する場合、
サービスごとに異なる{\PW}を利用することが望ましいが、
すべての{\PW}を記憶しておくことはほとんど不可能である。
%
Flor\^{e}ncioの...における調査では...\cite{Florencio:2007:LSW:1242572.1242661}、
また2011年の野村総研の調査によれば、
一般的なユーザが{\PW}認証を行なうサイトは平均19.4個で、
利用している{\PW}は平均3.1個であった\cite{野村総研}。
多数の{\PW}を記憶することが困難であるため、
多くのユーザが同じ{\PW}を複数サイトで使い回しているのだと思われる。

異なる{\PW}をすべて記憶することは不可能なのでどこかに記録しておく必要があるが、
{\PW}文字列をそのまま記録するのは危険なので、
複数の{\PW}を秘密情報として扱うための{\PW}管理システムが利用されている。
{\PW}管理システムは
1個の「マスター{\PW}」を利用してあらゆる{\PW}を管理するもので、
暗号化されたデータベースに{\PW}を格納するもの%
\cite{OnePassword}%
\cite{Dashlane}%
\cite{ミルパス}%
\cite{LastPass}%
\cite{KeyPass}%
\cite{NortonIDSafe}%
\cite{IDManager}%
が多いが、サービス名をもとにマスター{\PW}を変換することによって
複数の{\PW}を生成するシステム\cite{SuperGenPass}もある。
% 前者はデータベースを解読される危険があるのに対し、
% 後者は{\PW}本体をどこにも保存していないためより安全であるが、
両者ともにマスター{\PW}の記憶は必須であり、
マスター{\PW}を盗まれたり忘れたりする危険がある。
% {\PW}を毎日使わない人なら忘れてしまう可能性は高い。

ユーザは{\PW}を忘れてしまうことが非常に多いため、
多くのサービスにおいて{\PW}を復元したり初期化したりする手段が用意されている。
ユーザが{\SQ}に対する答を登録し、
質問に正しく回答することによって{\PW}を復元したりリセットできるサービスは多いし、
{\SQ}に答えることによって
{\PW}管理システムのマスター{\PW}を復元するシステム\cite{平野亮:2011-11-07}も提案されている。

新しく覚えた情報や新しく考えた情報はどうしても忘れてしまう可能性があるので、
新しく作成した{\PW}文字列を記憶して認証に利用することは本質的に無理がある。
一方、既知で忘れることがない{\EM}を{\SQ}として
認証のために直接利用することができれば、
認証に必要な情報を忘れてしまうことがないはずである。
多くの画像認証システム\cite{小池英樹:2006-05-15}は
{\SQ}に対して適切な操作を行なうことによって認証を行なっているため
{\PW}のようなものを記憶する必要がない。
%認証方法を忘れにくいという特長がある。
% {\SQ}が{\PW}と同じぐらい強力であれば、
% それをメインの認証手段にしてしまえばいいことになる。
%
画像認証システムはまだ普及しておらず利用できる環境は限られているが、
忘れない{\EM}を利用した{\SQ}への回答を強力な{\PW}に変換するシステムがあれば、
通常の{\PW}認証を用いた現在の様々なサービス上で、
認証方法を忘れる心配なく安全に認証を行なうことができるようになる。
本論文ではこのようなシステム「{\EP}」について述べる。

\section{{\EP}}

\subsection{{\EP}の原理}

{\EP}は、
ユーザが忘れることがない個人的な{\EM}を文字列に変換することによって
安全な{\PW}を生成するシステムである。
{\PW}文字列は以下の手順で生成される。

\begin{enumerate}
\item {\PW}生成の「種」となる文字列を用意する。
以下ではこれを「{\SS}」と表現する。
\item 忘れることがない個人的な{\EM}にもとづく{\SQ}を複数作成し、
それぞれについてひとつの正答と複数の偽答を用意する。
\item 質問と回答の組にもとづいて{\SS}に換字操作を行なう。
すべてに正しく回答したとき生成される文字列を{\PW}として利用する。
\end{enumerate}

\subsection{{\EP}利用例}

以下に{\EP}の利用例を示す。

\subsubsection{ブラウザでの利用}

筆者がtwitterの{\PW}を生成するために
ブラウザで{\EP}を利用している例を図\ref{web1}に示す\footnote{
  \textsf{http://{\EP}.com/masui}
}。
{\SS}として「\textsf{Twitter123456}」という文字列を指定しており、
4個の{\SQ}に対する回答選択に応じて
「\textsf{Mfveabn574923}」のような{\PW}候補が生成される。
異なる答をを選択すると全く異なる文字列が生成される。
{\SS}の8文字目が数字である場合は{\PW}の8文字目も数字になるなど、
{\SS}の文字種に対応した{\PW}候補が生成される。

最初の{\SQ}は筆者の小学校の同級生に関するもので、
最後の質問は数年前の体験に関するものである。
これらの質問は古い{\EM}にもとづいており、
筆者が将来答を忘れることはほとんど考えられないが、
本人以外がこのような質問に答えることは難しいので
正しい{\PW}を得ることはできない。

\begin{figure}[H]
\centerline{\includegraphics[width=83mm,bb=0 0 718 796]{figures/785ff09b4233804d2ec89c3af71ee5d0.png}}
\caption{ブラウザ上でTwitterの{\PW}を生成.}
\label{web1}
\end{figure}

{\SQ}と答はブラウザで編集でき、
右上の「サーバにセーブ」ボタンを押すことにより{\SS}、秘密の問題、答のリストがサーバにセーブされる。
「ファイルにセーブ」ボタンを押すとJSONデータをパソコンにダウンロードでき、
パソコン上のJSONデータをブラウザにドラッグ\&ドロップするとサーバにアップロードできる。
ユーザはどれが正答かを指定するわけではないので
ユーザの{\PW}はサーバにはわからない。

{\SS}を「\textsf{Facebook123456}」に変更すると、生成される{\PW}は図\ref{web2}のようにが変化する。
このように、サービスごとに異なる{\SS}を利用することによって
様々な{\PW}を簡単に生成できる。

\begin{figure}[H]
\centerline{\includegraphics[width=83mm,bb=0 0 718 265]{figures/36c371a13a8250c60fb9c03174382443.png}}
\caption{FaceBookの{\PW}を計算.}
\label{web2}
\end{figure}

{\PW}として大文字/小文字英数字と記号をすべて利用しなければならないサービスの場合は
{\SS}に「\textsf{PassWord123!@\#}」のような文字列を指定すればよい。

\subsubsection{回答選択によりパスワードを切り換え}

サービスごとに{\SS}を変えるのではなく、
図\ref{web3}のように
サービス名に応じて回答を変えることによって異なる{\PW}を作成することも可能である。
{\PW}としての長さや文字種に特種な制約が無いサービスに関してはこの方法が便利である。

\begin{figure}[H]
\centerline{\includegraphics[width=83mm,bb=0 0 718 428]{figures/a9167a6ec6af9c70dd1617e3fc25ec30.png}}
\centerline{\includegraphics[width=83mm,bb=0 0 718 428]{figures/5b887fabeb8e3319623901fe4a6c56f2.png}}
\caption{サービス名に対応する回答で異なる{\PW}を生成.}
\label{web3}
\end{figure}

定期的に{\PW}変更を求められるシステムでは、
「利用期間は?」という質問に対して
「2013/1」「2013/2」のような答を用意しておけば、
時期を選択することによって簡単に{\PW}を変更することができるので便利である。

\subsubsection{既存{\PW}の利用}

現在利用している{\PW}(e.g. ``\textsf{Masui1234}'')を利用したい場合、
図\ref{web4}のように{\PW}欄に現在の{\PW}を入力すれば
それを生成する{\SS}が自動生成されるので、
生成された{\SS}を記録しておけばよい。

\begin{figure}[H]
\centerline{\includegraphics[width=83mm,bb=0 0 718 479]{figures/fab9c55242e1d52c89ff1b46d77b3168.png}}
\caption{既存の{\PW}を利用.}
\label{web4}
\end{figure}

{\PW}以外の秘密の文字列も管理することができる。
たとえば自転車の鍵番号「\textsf{1234}」を{\EP}で管理したい場合、
「\textsf{jitensha-0000}」という{\SS}を入力した後で
{\PW}枠の数字を「\textsf{1234}」に変更することにより、
図\ref{web5}のように「\textsf{jitensha-6145}」という{\SS}で
鍵番号を管理できるようになる。

\begin{figure}[H]
\centerline{\includegraphics[width=83mm,bb=0 0 718 479]{figures/494cef6da1be0a069ee56e7ec8dcb9a7.png}}
\caption{既存の{\PW}を利用.}
\label{web5}
\end{figure}

\subsubsection{Androidアプリ}

Webサービスを利用する場合、ブラウザとサーバとの間の通信を
記録されたり盗み見されたりされる危険が皆無ではない。
前述の例において、
{\PW}はブラウザ内部でJavaScriptにより生成されるので、
一度ページを表示した後は
ネットワークを遮断しても{\PW}計算を行なえるようになっているが、
最初から全く通信を行なわずに{\PW}を作成できる方がより安心であろう。
このため、通信を全く行なわずにマシン単体で{\PW}計算を行なうための
Androidアプリを用意した。
ブラウザの右上の「Androidアプリ」ボタンを押すと、
現在表示している秘密の問題と答を内蔵したAndroidアプリが
サーバ上でビルドされてダウンロード可能になる。

Android端末でアプリを実行すると図\ref{android1}のような画面が表示される。
{\SS}を設定して「開始」ボタンを押すと図\ref{android2}のように質問がひとつずつ表示され、
すべて回答すると{\PW}が計算され図\ref{android3}のように表示される。

\begin{figure}[H]
\centerline{\includegraphics[width=50mm,bb=0 0 720 720]{figures/android1crop.png}}
\caption{Androidアプリ初期画面.}
\label{android1}
\end{figure}

\begin{figure}[H]
\centerline{\includegraphics[width=50mm,bb=0 0 720 900]{figures/android2crop.png}}
\caption{最初の問題.}
\label{android2}
\end{figure}

\begin{figure}[H]
\centerline{\includegraphics[width=50mm,bb=0 0 720 400]{figures/android3crop.png}}
\caption{すべての問題に回答した後の状態.}
\label{android3}
\end{figure}

回答入力と{\PW}計算はAndroid端末で実行されるため、
端末を「機内モード」に設定するなどの方法で
あらゆるネットワーク接続を遮断した状態でも{\PW}を計算することができる。
{\EP}をインストールしたAndroid端末を持っていれば常に各種の{\PW}を計算できるので、
他人のマシンや公共の場所に設置されたパソコンなどでも
安心してtwitterなどのネットサービスを利用することができる。

前述の方法で{\EP}アプリをサーバからダウンロードする場合は、
ブラウザ上で秘密の問題をサーバに登録する必要があるが、
秘密の問題を全くネット上に露出することなくアプリを利用することもできる。
秘密の問題を含まない{\EP}アプリをGoogle Playで公開\footnote{
 {\textsf{https://play.google.com/{\allowbreak}store/{\allowbreak}apps/{\allowbreak}details?{\allowbreak}id=com.{\allowbreak}pitecan.{\allowbreak}episo\_pass}}
}しているので、
これを端末にインストールした後、
ローカルマシンで作成した{\SQ}を端末に転送すれば
EpisoPass.comからダウンロードしたアプリと同様に利用できる。
この手法を使うと{\SQ}が通信路を通ることがないので安全であるが、
アプリのセットアップの手間は増える。

\subsection{{\PW}の計算方法}

問題と回答から文字列を生成し、そのMD5値によって{\SS}を換字することにより
{\PW}を生成している。
{\PW}文字列の計算方法は附録に示す。

\section{議論}

\subsection{運用形態と安全性}

{\PW}管理システムで最も重要なのは安全性である。
%
{\EP}は様々な運用方法が可能であり、
運用形態によって安全性の評価が異なるので、
いくつかの利用例に対して{\EP}の安全性を考える。

\subsubsection{\protect{\textsf{\EP}}の利用を秘密にする場合}
\label{pattern1}

{\PW}管理に{\EP}を利用していることを公開せず、
{\PW}生成機として利用するだけの場合は、
普通に{\PW}を利用する場合に比べて安全度が低下することは無い。
{\PW}として充分強力な文字列を{\SS}として利用すれば
{\EP}によって換字された文字列も{\PW}として強力だと考えられるし、
推測しやすい文字列を{\SS}として設定した場合であっても
ランダムな換字操作によってより強力な{\PW}が生成されるので、
{\EP}を利用するデメリットは無い。

\subsubsection{{\SS}を秘密にする場合}
\label{pattern2}

{\EP}の秘密の問題を公開した場合でも、
{\SS}を通常の{\PW}と同じレベルで秘密にしておけば
SuperGenPassと\cite{SuperGenPass}と同じレベルの安全性は確保できるはずであるが、
計算された{\PW}のひとつと秘密の問題の答が共に流出してしまった場合
逆計算によって{\SS}が判明してしまう可能性がある。
{\PW}の流出に普通に注意しつつ、
かつ他人にはわからない秘密の問題を複数利用している場合は
このような問題が起こる可能性は極めて低いと考えられる。

% しかし{\EP}で計算された{\PW}のひとつが他人に

\subsubsection{すべて公開する場合}

{\SQ}をすべて解くことが不可能であれば、
{\SS}と{\SQ}をすべて公開しても安全である。
この場合、
ユーザは{\SQ}と{\SS}を通常のテキストデータと同じように管理できるし、
新たに記憶しなければならない情報が不要なので、
ユーザはパスワード管理について注意を払う必要が無いので気が楽である。

一方、
{\SQ}を解くことを困難にするためには質問の数と偽答の数が多くなければならないし、
自分だけが知っている{\EM}をうまく{\SQ}にするにはコツが必要である。
すべての質問を公開するのが心配な場合や、
{\SQ}の数や質がが充分でないと感じられる場合は
\ref{pattern1}や\ref{pattern2}の手法で運用し、
徐々に運用方法を変えていけば良いだろう。

\subsection{{\SQ}に関する議論}

認証に{\SQ}を利用する方法は現在

選択式のとそうでないものと

すべての計算結果を他人に見せなければ総当たり攻撃は回避できる。

{\SQ}を解くことによって{\PW}をリセットできる機能があれば
{\PW}を忘れても大丈夫だという利点はあるものの、
このような方式がシステムを脆弱にしていることが近年問題になっている\footnote{
  ..... どこかの記者
}。

簡単な質問でリセットできるのが問題になっている。

{\SQ}は実は忘れることが多いという問題があるらしい。

平野\cite{平野亮:2011-11-07}は
{\SQ}からマスター{\PW}を復元する方法を提案している。
Shamirのなんたら方式を使い、全部正解しなくても復元できるのがミソ。

{\PW}は自分で考えることを想定しているのかもしれない?

% \textbf{どれが正解か知っていなければこの方法は適用できないのでは?}

{\EP}は、
{\PW}を忘れたときに{\SQ}を利用するのではなく、
{\PW}の生成に{\SQ}を利用するという点が異なっている。

兼子は、
{\SQ}のエントロピを上げる方法を提案している\cite{Kaneko}。

多くのシステムにおいて、
「母親の旧姓は?」や「最初に飼ったペットの名前は?」のような、
あらかじめ決まった質問のみを利用できるようになっているが、
このような問題は他人が解くことも容易であるうえに、
{\SQ}の数は少ないのが普通だからである\cite{Rabkin:2008:PKQ:1408664.1408667}。
% 銀行20個調べて{\SQ}の弱さを指摘

{\EP}はシステムに用意された問題を利用せず、
ユーザが自分で考えたなぞなぞ問題を利用するので、
他人には解くことが難しく自分では忘れない問題を自由に作成できる。
これは強力なはずであるが、
そのような問題を作成することは難しいということが
知られている\cite{Just:2009:PCC:1572532.1572543}\cite{Schechter:2009:NSM:1607723.1608145}。
%
%  {\SQ}の問題
%   自分で[[{\SQ}]]を決めるととても弱くなることが判明
%   複数の{\SQ}を使うとマシになる?
%   結構人は間違うものらしい
%  applicability or repeatability の問題がある
%   Applicability: How widely applicable is the given question?
%   Memorability: How easy is it for the user to recall the answer?
%   Repeatability: How accurately can the answer be replayed, without syntactic or semantic ambiguity?
%  ユーザに質問を作らせると全然駄目なことが多い
%   すぐ解けてしまうもの
%   思い出せないもの
%
% 自分が作るものでくだらない{\SQ}は駄目
%
古い記憶にもとづいて作成した秘密の問題はユーザが想像するよりも解かれやすい。
この問題を避けるため、問題と答を連想するために画像を利用したり、
複数の問題を連続的に利用したりする手法も提案されている\cite{Renaud:2010:PQE:2146303.2146318}。

%  昔の記憶にもとづく[[{\SQ}]]はアタックされやすい
%  質問に写真を加えることにより思い出しやすくなったり
%  解決策
%   答を思い出すために画像を使う
%    動物や場所の写真から人を思い出すなど
%   必ずしも本当でない連想を使う
%   連続的に出題する

% {\SQ}はシステムに指定される場合とユーザが作成できる場合がある。
% システムが質問を提供する場合、あらゆるユーザが答えられる質問にしなければならないため、
% 「ペットの名前は?」「母親の旧姓は?」
% のような簡単なものであることが多いが、
% このような質問は他人が解くことも容易であるうえに、
% 質問の数も充分でないことが多いため充分安全でない\cite{Rabkin:2008:PKQ:1408664.1408667}。
% %
% ユーザが質問を作成できる場合、
% 他人に解くことが難しく自分では忘れない問題を自由に作成できるはずであるが、
% そのような問題を作成することは難しいということが
% 知られている\cite{Just:2009:PCC:1572532.1572543}\cite{Schechter:2009:NSM:1607723.1608145}。

\subsection{総当たり攻撃への耐性}

\subsection{{\EP}の利点}

\paragraph{忘れないこと}

忘れない問題を作ればパスワードを忘れることがない
(すでに持っている忘れない知識をもとにしてパスワードを生成できる)
 = 新たに何かを覚える必要が無いのが良い

\paragraph{オープンなパスワード管理}

普通のテキストファイルとしてパスワード情報を管理できる

滅多に使わないサービスのパスワードなどもリストしておけばOK

\paragraph{パスワードの管理と作成が両方できる}

SuperGenPass\cite{SuperGenPass}は良いけれども
既存の{\PW}を登録することができない。

{\EP}は新しいものの生成も既存のものの記憶も両方できる

\paragraph{手軽で柔軟な運用}

グループで運用したりできる

選択肢を増やせばいくらでも強力にできる

\subsection{{\EP}の欠点}

脆弱性の可能性

なぞなぞ問題を考えるのが面倒

アプリに問題を登録するのが面倒

心配に感じる点
 => 自分が知ってることは他人も知ってそうに思える

サーバに情報を抜かれるかもしれないと思う
 => そもそもEpisoPassを使ってることを隠せばいいのだが

\subsection{簡単か?}

原理は簡単で、ソースを全公開しているので
問題があれば指摘されることが期待できる

\subsection{偽答を考える方法}

{\SQ}を考えるのがとにかく面倒

ライバルサーチ「やーやら」
同位語検索エンジン\cite{大島裕明:2006-12-15}

BooWa
\cite{Wang:2007:LSE:1441428.1442086}
Boo!Wa!というサイトをやっている\cite{BooWa}

\cite{Huang:2012:LFC:2426725.2426728}読んでないけど

人名/地名/会社名などは似たようなものをすぐ作れる

偽答を自動生成することは可能だが、
偽答の名前リストと同級生の名前リストのandを取ると答がわかるかもしれない

こういう危険は沢山存在するし、
ソーシャルエンジニアリング\cite{ソーシャル}にひっかかる可能性も高いから
完全に安全なものは難しいかもしれない

\subsection{どういう質問が良いか}

一般的にシステムから与えられるのはひどいのが多い
母親の旧姓とかペットの名前とかすぐわかる

古いどうでもよいエピソード記憶がいい

自慢にならないもの
  自慢はしゃべってしまうかも
どちらかというとネガティブなもの
  何かまけたとか怪我したとか
好きとか嫌いとかいうのは他人からわかるし変わるかもしれないので事実がいいと思う

学生に作ってもらった例
  みんなで行った店は
  好きな言葉は

デルタの柵の上にあるのは?


\subsection{画像の利用}

画像認証がイイと言ってる人も多い

画像も使えるよ

Leiko氏のなぞなぞ画像なんかどうか

\subsection{他者が答を知っているときの危険}

自転車のカギ番号を共有しているとき
それを{\EP}で管理すると全質問の答がわかってしまう

家族のなぞなぞ問題を作るといった工夫

\subsection{自由記述にしない理由}

{\EP}では{\SQ}の答をリストから選択する方法をとっているが、
{\SQ}によって{\PW}を回復できるWebサービスなどでは
ユーザが回答文字列を入力する方法をとっているものが多い。
リストから答を選択するよりも文字列を入力する方が安全そうに思えるが、
ターミナルでのコマンドライン入力がGUIのメニュー操作より難しいのと同様に、
入力するテキストを忘れたり間違えたりする可能性が高くなってしまう。
「日本の首都は?」のような簡単な質問であっても
「東京」「Tokyo」「tokyo」「toukyou」のどれが正解か迷うかもしれない。
自由なテキスト入力よりも、
選択式の回答を多数用意して提示する方が高速で誤りのない操作が可能である。


\subsection{{\EM}をいろんなものに変換する面白さ}

{\PW}に限らないだろう

\subsection{指紋を{\PW}に変換することなども考えられる}

指紋はコピーされる

脳はコピーされない

同じ考え方は通用するだろう

\subsection{安全性}

エントロピー

\subsection{安心感}

知ってる話ばかりだと心配になる

\section{結論}

{\EP}は\textsf{http://Episopass.com/}で運用しており、
ソースはGitHubで公開している\footnote{
  \textsf{http://GitHub.com/masui/EpisoPass}
}。

\bibliographystyle{jwiss}
\bibliography{paper}

\end{document}
