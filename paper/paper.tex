\documentclass[twoside]{wiss}

%\usepackage{ascmac}
%\usepackage[dvips]{graphicx}

\journalhead{EpisoPass: Holy Grail}

\begin{document}

% Smooth & Snapping 
% SnapSelector
% SmoothSnap

\title{EpisoPass: すごいパスワード}
\etitle{EpisoPass: Holy Grail of Password}

\author{増井俊之\affil{Toshiyuki Masui, 慶應義塾大学 環境情報学部}}

\begin{abstract}
EpisoPass EpisoPass EpisoPass EpisoPass EpisoPass EpisoPass 
EpisoPass EpisoPass EpisoPass EpisoPass EpisoPass EpisoPass 
EpisoPass EpisoPass EpisoPass EpisoPass EpisoPass EpisoPass 
EpisoPass EpisoPass EpisoPass EpisoPass EpisoPass EpisoPass 
\end{abstract}

\maketitle

\section{はじめに}

 銀行20個調べて秘密の質問の弱さを指摘
\cite{Rabkin:2008:PKQ:1408664.1408667}

 秘密の質問の問題
  自分で[[秘密の質問]]を決めるととても弱くなることが判明
  複数の秘密の質問を使うとマシになる?
  結構人は間違うものらしい
 applicability or repeatability の問題がある
  Applicability: How widely applicable is the given question?
  Memorability: How easy is it for the user to recall the answer?
  Repeatability: How accurately can the answer be replayed, without syntactic or semantic ambiguity?
 ユーザに質問を作らせると全然駄目なことが多い
  すぐ解けてしまうもの
  思い出せないもの
\cite{Just:2009:PCC:1572532.1572543}

 昔の記憶にもとづく[[秘密の質問]]はアタックされやすい
 質問に写真を加えることにより思い出しやすくなったり
 解決策
  答を思い出すために画像を使う
   動物や場所の写真から人を思い出すなど
  必ずしも本当でない連想を使う
  連続的に出題する
\cite{Renaud:2010:PQE:2146303.2146318}

くだらない秘密の質問は駄目\cite{Schechter:2009:NSM:1607723.1608145}

\bibliographystyle{jwiss}
\bibliography{paper}

\end{document}
