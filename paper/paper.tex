\documentclass[twoside]{wiss}

%\usepackage{ascmac}
%\usepackage[dvips]{graphicx}

\journalhead{EpisoPass: Holy Grail}

\begin{document}

% Smooth & Snapping 
% SnapSelector
% SmoothSnap

\title{EpisoPass: すごいパスワード}
\etitle{EpisoPass: Holy Grail of Password}

\author{増井俊之\affil{Toshiyuki Masui, 慶應義塾大学 環境情報学部}}

\begin{abstract}
EpisoPass EpisoPass EpisoPass EpisoPass EpisoPass EpisoPass 
EpisoPass EpisoPass EpisoPass EpisoPass EpisoPass EpisoPass 
EpisoPass EpisoPass EpisoPass EpisoPass EpisoPass EpisoPass 
EpisoPass EpisoPass EpisoPass EpisoPass EpisoPass EpisoPass 
\end{abstract}

\maketitle

\section{はじめに}

個人認証のために現在パスワードが広く利用されている。
パスワードによる認証には多くの問題があるが\cite{xxxxx}、
今後も長い間利用され続けることは間違いない。

パスワードは忘れるものである。
\cite{Florencio:2007:LSW:1242572.1242661}
\cite{野村総研}
これは重要な問題のひとつである

同じパスワードの使い回しは危険だが、
ちゃんと運用している人はほとんどいないことがわかる。

一括管理システムは沢山ある
\cite{キングジム}\cite{LastPass}
マスターパスワードを覚えなければならないし
マスターパスワードを盗まれると困る
パスワードを毎日使わない人なら忘れてしまう可能性は高い。

マスターパスワードを変換する逐次生成システムもある\cite{SuperGenPass}。
パスワードをどこにも保存していないので安全であるが、
こちらもマスターパスワードを忘れると困る。

ユーザが設定したパスワードを忘れてしまうことは非常に多いため、
ユーザが秘密の質問\footnote{
  ``Challenge Question'', ``Secret Question''と呼ばれる
}と答を登録し、
質問に正しく回答するとパスワードをリセットできるようになっているWebサービスが多い。

パスワードを考えたり覚えたりすることが難しいのがいけない

どんなものでも新しく考えたものは忘れるのがあたりまえなので本質的に問題がある

\begin{itemize}
\item 新しく何かを覚えず、知ってるものからパスワードを生成する
\end{itemize}

エピソード知識を利用する!

エピソード記憶とは...

\section{EpisoPass}

EpisoPassは、
ユーザが忘れることがない個人的なエピソード記憶を文字列に変換することによって
安全なパスワードを生成するシステムである。
パスワード文字列は以下の手順で生成される。

\begin{enumerate}
\item パスワード生成の「種」となる文字列を用意する。
以下ではこれを「シード文字列」と表現する。
\item 忘れることがない個人的なエピソード記憶にもとづく秘密の質問を複数作成し、
それぞれについてひとつの正答と複数の偽答を用意する。
\item 質問と回答の組にもとづいてシード文字列に換字操作を行なう。
すべてに正しく回答したとき生成される文字列をパスワードとして利用する。
\end{enumerate}


自分だけが知っていて忘れることがないエピソード知識からパスワードを生成するシステムである

(特長)

\subsection{EpisoPassの利用例}

\subsection{画像の利用}

画像認証がイイと言ってる人も多い

画像も使えるよ

\section{考察}

\section{エピソード記憶をいろんなものに変換する面白さ}

\section{安全性}

エントロピー

\subsection{秘密の質問の脆弱性}

秘密の質問を解くことによってパスワードをリセットできる機能があれば
パスワードを忘れても大丈夫だという利点はあるものの、
このような方式がシステムを脆弱にしていることが近年問題になっている\footnote{
  ..... どこかの記者
}。

簡単な質問でリセットできるのが問題になっている。

秘密の質問は実は忘れることが多いという問題があるらしい。

平野\cite{平野亮:2011-11-07}は
秘密の質問からマスターパスワードを復元する方法を提案している。
Shamirのなんたら方式を使い、全部正解しなくても復元できるのがミソ。
%
パスワードは自分で考えることを想定しているのかもしれない?
%
\textbf{どれが正解か知っていなければこの方法は適用できないのでは?}

EpisoPassは、
パスワードを忘れたときに秘密の質問を利用するのではなく、
パスワードの生成に秘密の質問を利用するという点が異なっている。

兼子は、
秘密の質問のエントロピを上げる方法を提案している\cite{Kaneko}。

多くのシステムにおいて、
「母親の旧姓は?」や「最初に飼ったペットの名前は?」のような、
あらかじめ決まった質問のみを利用できるようになっているが、
このような問題は他人が解くことも容易であるうえに、
秘密の質問の数は少ないのが普通だからである\cite{Rabkin:2008:PKQ:1408664.1408667}。
% 銀行20個調べて秘密の質問の弱さを指摘

EpisoPassはシステムに用意された問題を利用せず、
ユーザが自分で考えたなぞなぞ問題を利用するので、
他人には解くことが難しく自分では忘れない問題を自由に作成できる。
これは強力なはずであるが、
そのような問題を作成することは難しいということが
知られている\cite{Just:2009:PCC:1572532.1572543}\cite{Schechter:2009:NSM:1607723.1608145}。
%
%  秘密の質問の問題
%   自分で[[秘密の質問]]を決めるととても弱くなることが判明
%   複数の秘密の質問を使うとマシになる?
%   結構人は間違うものらしい
%  applicability or repeatability の問題がある
%   Applicability: How widely applicable is the given question?
%   Memorability: How easy is it for the user to recall the answer?
%   Repeatability: How accurately can the answer be replayed, without syntactic or semantic ambiguity?
%  ユーザに質問を作らせると全然駄目なことが多い
%   すぐ解けてしまうもの
%   思い出せないもの
%
% 自分が作るものでくだらない秘密の質問は駄目
%
古い記憶にもとづいて作成した秘密の問題はユーザが想像するよりも解かれやすい。
この問題を避けるため、問題と答を連想するために画像を利用したり、
複数の問題を連続的に利用したりする手法も提案されている\cite{Renaud:2010:PQE:2146303.2146318}。

%  昔の記憶にもとづく[[秘密の質問]]はアタックされやすい
%  質問に写真を加えることにより思い出しやすくなったり
%  解決策
%   答を思い出すために画像を使う
%    動物や場所の写真から人を思い出すなど
%   必ずしも本当でない連想を使う
%   連続的に出題する

\bibliographystyle{jwiss}
\bibliography{paper}

\end{document}
